\documentclass[10pt, a4paper]{article}
\usepackage{amsmath}
\usepackage{amssymb}
\usepackage{amsfonts}
\usepackage{amsthm}
\usepackage{mathrsfs}
\usepackage{mathtools}
\usepackage{bbm}
\usepackage{listings}
\usepackage{enumitem} % reference enum item
\usepackage{wasysym} % smiley faces!

% COLORS
\usepackage{color}
\usepackage{xcolor}
\definecolor{mygray}{rgb}{0.4,0.4,0.4}

% Math Utils
\newcommand{\inner}[2]{\langle#1,\,#2\rangle}
\newcommand{\innerone}[1]{\langle#1\rangle}
\newcommand{\paren}[1]{\left(#1\right)}
\newcommand{\bracket}[1]{\left[#1\right]}
\newcommand{\given}{\biggr\rvert}
\newcommand{\PP}[1]{\mathbb{P}\paren{#1}}
\newcommand{\PPB}[1]{\mathbb{P}\bracket{#1}}
\newcommand{\PPP}{\mathbb{P}}
\newcommand{\E}{\mathbb{E}}
\newcommand{\EE}[1]{\mathbb{E}\paren{#1}}
\newcommand{\EEB}[1]{\mathbb{E}\bracket{#1}}
\newcommand{\Tr}[1]{\text{Tr}\paren{#1}}
\newcommand{\R}{\mathbb{R}}
\newcommand{\Z}{\mathbb{Z}}
\newcommand{\Q}{\mathbb{Q}}
\newcommand{\N}{\mathbb{N}}
\newcommand{\xeq}[1]{\stackrel{#1}{=}}
\newcommand{\indist}{\mathcal{D}}
\newcommand{\inprob}{\mathcal{P}}
\newcommand{\borel}{\mathcal{B}}
\newcommand{\indic}[1]{\mathbbm{1}_{#1}}
\newcommand\restr[2]{{% we make the whole thing an ordinary symbol
 \left.\kern-\nulldelimiterspace% automatically resize the bar with \right
 #1 % the function
 \vphantom{\big|} % pretend it's a little taller at normal size
 \right|_{#2} % this is the delimiter
 }}
\newcommand{\im}{\text{Im}}
\newcommand{\cov}[2]{\text{Cov}\paren{#1,#2}}
\newcommand{\corr}[2]{\text{Corr}\paren{#1,#2}}
\newcommand{\var}[1]{\text{Var}\paren{#1}}
\newcommand{\abs}[1]{\left\lvert#1\right\rvert}
\newcommand{\norm}[1]{\vert\vert#1\rvert\rvert}
\newcommand{\grad}{\nabla}
\newcommand{\varrow}{\overset{\rightharpoonup}}
\newcommand{\diag}[1]{\text{diag}\paren{#1}}
\newcommand{\mesh}{\text{mesh}}
\newcommand{\sgn}[1]{\text{sgn}\paren{#1}}
\newcommand{\set}[1]{\{#1\}}
\newcommand{\swap}{\operatorname{swap}}

\DeclareMathOperator*{\argmax}{arg\,max}
\DeclareMathOperator*{\argmin}{arg\,min}
\DeclareMathOperator*{\tv}{d_{TV}}

\DeclarePairedDelimiter{\floor}{\lfloor}{\rfloor}
\DeclarePairedDelimiter{\ceil}{\lceil}{\rceil}

\newcommand\indep{\protect\mathpalette{\protect\independenT}{\perp}}
\def\independenT#1#2{\mathrel{\rlap{$#1#2$}\mkern2mu{#1#2}}}

\newcommand{\code}[1]{{\texttt{\textcolor{mygray}{\small #1}}}}

% TODO highlighting 
\newcommand{\todo}[1]{\textcolor{red}{#1}}

% Theorem stuff
\newtheorem{theorem}{Theorem}[section]
\newtheorem{definition}[theorem]{Definition}
\newtheorem{corollary}{Corollary}[theorem]
\newtheorem{lemma}[theorem]{Lemma}
\newtheorem{exercise}{Exercise}[section]
\newtheorem{remark}{Remark}[section]
\newtheorem{example}{Example}[section]

\usepackage{cite}
\usepackage{fullpage}

\begin{document}
\title{Exponential Model}
\author{James Yang}
\maketitle

\section{Introduction}

The exponential model is one where each arm $i$ is assumed to follow
an exponential distribution with hazard $\lambda_i$.
Depending on the design procedure (the test statistic)
there are numerous choices of grid spaces and parametrizations
of the natural parameter space.
This document will focus on the log-rank statistic with two arms 
(control and a treatment).

\section{Model Assumptions}\label{sec:model}

Assume that there are $n$ patients in each of the $d=2$ arms
with independent $X_{ci} \sim E(\lambda_c), X_{ti} \sim E(\lambda_t)$,
$i=1,\ldots, n$.
$X_{c\cdot}$ are the samples for the control arm and
$X_{t\cdot}$ are for the treatment arm.
Then, the distribution of $X \in \R^{n \times 2}$ forms an exponential family
with sufficient statistic $T(x) = \paren{\sum\limits_{i=1}^n x_{ci}, \sum\limits_{i=1}^n x_{ti}}$
natural parameter $\eta = (-\lambda_c, -\lambda_t)$,
and log-partition function $A(\eta) := -n\log(\eta_c \eta_t)$.

\section{Grid Space}

Since the log-rank statistic only depends on the hazard rate
$h = \lambda_t / \lambda_c$,
it is convenient to parametrize the natural parameter space
as a function of $(\lambda_c, h)$.
Moreover, we will see in Section~\ref{ssec:max-cov-quad-form}
that we get major computation benefits of parametrizing in the log-space
$\theta = (\log(\lambda_c), \log(h))$.

This parametrization defines a mapping $\eta(\theta) = \paren{-e^{\theta_1}, -e^{\theta_1+\theta_2}}$
from the grid space to the natural parameter space.
We conclude this section with the Jacobian and hessian computations
needed in the later sections.
\begin{align}
    D_\theta \eta(\theta)
    &=
    \begin{bmatrix}
        -e^{\theta_1} & 0 \\
        -e^{\theta_1+\theta_2} & -e^{\theta_1+\theta_2}
    \end{bmatrix}
    \label{eq:eta-jac}
    \\
    \nabla^2_\theta \eta_1(\theta)
    &=
    -e^{\theta_1} e_1e_1^\top 
    \label{eq:eta-1-hess}
    \\
    \nabla^2_\theta \eta_2(\theta)
    &=
    -e^{\theta_1+\theta_2} \vec{1} \vec{1}^\top
    \label{eq:eta-2-hess}
\end{align}
where $ e_i$ is the ith standard basis vector 
and $ \vec{1}$ is a vector of ones.

\section{Upper Bound}

For any model, we must be able to compute
the upper bound estimate.
The generalized upper bound estimate requires model-specific quantities,
which are given by
\begin{align*}
    \text{Gradient Term}&: T(x) - \nabla_\eta A(\eta) \\
    \text{$\eta$ transform}&: D_\theta\eta(\theta) v \\
    \text{Covariance quadratic form}&: u^\top \var{T}_{\eta} u \\
    \text{Max covariance quadratic form}&:
    \sup\limits_{\theta \in R} \bracket{v^\top (D\eta(\theta))^\top \var{T}_{\eta(\theta)} (D\eta(\theta)) v} \\
    \text{Max covariance and $\eta$ hessian}&: 
    \norm{v}^2
    \sum\limits_{k=1}^d
    \sup\limits_{\theta \in R}
    \bracket{%
        \norm{\nabla^2 \eta_k(\theta)}_{op}
        \sqrt{\var{T_k}_{\eta(\theta)}}
    }
\end{align*}
for any $v, u \in \R^d$ and a bounded subset $R \subseteq \R^d$.

The next few subsections will derive the formulas
for each of the quantities above.

\subsection{Gradient Term}

As shown in Section~\ref{sec:model},
we have the form for $T(x)$ and $A(\eta)$.
\begin{align*}
    \nabla_\eta A(\eta)
    &=
    -n \paren{\eta_c^{-1}, \eta_t^{-1}}
    =
    n \paren{\lambda_c^{-1}, \lambda_t^{-1}}
\end{align*}
This gives us
\begin{align*}
    T(x) - \nabla_\eta A(\eta)
    &=
    \paren{%
        \sum\limits_{i=1}^n x_{ci} - n\lambda_c^{-1},
        \sum\limits_{i=1}^n x_{ti} - n\lambda_t^{-1}
    }
\end{align*}

\subsection{$\eta$ Transform}

Using Eq.~\ref{eq:eta-jac},
for any $v \in \R^d$,
\begin{align*}
    D_\theta \eta(\theta) v 
    &=
    -
    \begin{bmatrix}
        e^{\theta_1} & 0 \\
        e^{\theta_1+\theta_2} & e^{\theta_1+\theta_2} 
    \end{bmatrix}
    v
    =
    -
    \begin{bmatrix}
        e^{\theta_1} v_1 \\
        e^{\theta_1+\theta_2} (v_1+v_2)
    \end{bmatrix}
    =
    -
    \begin{bmatrix}
        \lambda_c v_1 \\
        \lambda_t (v_1+v_2)
    \end{bmatrix}
\end{align*}

\subsection{Covariance Quadratic Form}

The covariance of $T$ is given by 
\begin{align}
    \var{T}_\eta &= 
    n
    \begin{bmatrix}
        \eta_c^{-2} & 0 \\
        0 & \eta_t^{-2}
    \end{bmatrix}
    =
    n
    \begin{bmatrix}
        \lambda_c^{-2} & 0 \\
        0 & \lambda_t^{-2}
    \end{bmatrix}
    \label{eq:t-cov}
\end{align}
and so,
\begin{align*}
    u^\top \var{T}_\eta u &=
    n (u_1^2 \lambda_c^{-2} + u_2^2 \lambda_t^{-2})
\end{align*}

\subsection{Max Covariance Quadratic Form}\label{ssec:max-cov-quad-form}

Using Eq.~\ref{eq:eta-jac},~\ref{eq:t-cov},
\begin{align*}
    D_\theta \eta(\theta)^\top 
    \var{T}_\eta 
    D_\theta \eta(\theta)
    &=
    n
    D_\theta \eta(\theta)^\top
    \begin{bmatrix}
        \eta_c^{-2} & 0 \\
        0 & \eta_t^{-2}
    \end{bmatrix}
    \begin{bmatrix}
        e^{\theta_1} & 0 \\
        e^{\theta_1+\theta_2} & e^{\theta_1+\theta_2}
    \end{bmatrix}
    \\&=
    n
    D_\theta \eta(\theta)^\top
    \begin{bmatrix}
        e^{-2\theta_1} & 0 \\
        0 & e^{-2(\theta_1+\theta_2)}
    \end{bmatrix}
    \begin{bmatrix}
        e^{\theta_1} & 0 \\
        e^{\theta_1+\theta_2} & e^{\theta_1+\theta_2}
    \end{bmatrix}
    \\&=
    n
    \begin{bmatrix}
        e^{\theta_1} & e^{\theta_1+\theta_2} \\
        0 & e^{\theta_1+\theta_2}
    \end{bmatrix}
    \begin{bmatrix}
        e^{-\theta_1} & 0 \\
        e^{-(\theta_1+\theta_2)} & e^{-(\theta_1+\theta_2)}
    \end{bmatrix}
    \\&=
    n
    \begin{bmatrix}
        2 & 1 \\
        1 & 1
    \end{bmatrix}
\end{align*}
Note the incredible simplification due to our choice of the $\eta$ transformation.
This gives us
\begin{align*}
    \sup\limits_{\theta \in R} \bracket{%
    v^\top
        D_\theta \eta(\theta)^\top 
        \var{T}_\eta 
        D_\theta \eta(\theta)
    v
    }
    &=
    n v^\top
    \begin{bmatrix}
        2 & 1 \\
        1 & 1
    \end{bmatrix}
    v
\end{align*}

\subsection{Max Covariance and $\eta$ Hessian}

From Eq.~\ref{eq:eta-1-hess},~\ref{eq:eta-2-hess},
\begin{align*}
    \norm{\nabla^2 \eta_1(\theta)}_{op}
    &=
    e^{\theta_1} \norm{e_1 e_1^\top}_{op}
    =
    e^{\theta_1}
    \\
    \norm{\nabla^2 \eta_2(\theta)}_{op}
    &=
    e^{\theta_1+\theta_2} \norm{\vec{1}\vec{1}^\top}_{op}
    =
    e^{\theta_1+\theta_2} d
\end{align*}

This gives us
\begin{align*}
    \norm{v}^2
    \sum\limits_{k=1}^d
    \sup\limits_{\theta \in R}
    \bracket{%
        \norm{\nabla^2 \eta_k(\theta)}_{op}
        \sqrt{\var{T_k}_{\eta(\theta)}}
    }
    &=
    \norm{v}^2
    \sqrt{n}
    \paren{%
        1 + d
    }
\end{align*}
     

\end{document}
